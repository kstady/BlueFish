\documentclass{ximera}



\title{Exponents and Positive Powers of 10}   % Use xmlatex all -s to compile when SERVE refuses to work.
\author{Kelly Stady}
\license{CC: 0}         % replace with an appropriate license, or set it in xmPreamble

\begin{document}
\begin{abstract}

\end{abstract}
\maketitle
\label{xim:ExponentsPowers10Activity}


\section*{Exponential Notation}

Exponential notation is shorthand for repeated multiplication of
the same quantity. For example, the product $5 \cdot 5 \cdot 5$ can be rewritten more simply
as $5^3.$  In this example, the number 5 is called the
\textbf{base} and the number 3 is called the \textbf{exponent}. In
general, the exponent indicates the number of times to multiply
the base.


\begin{problem} Identify the base and exponent of $10^7.$


\hspace{2in} \textbf{Base} $\begin{prompt}=\answer{10}\end{prompt}$  \hspace{1in}  \textbf{Exponent} $\begin{prompt}=\answer{7}\end{prompt}$

\end{problem}


\begin{problem} Identify the base and exponent of $4^5.$


\hspace{2in} \textbf{Base} $\begin{prompt}=\answer{4}\end{prompt}$  \hspace{1in}  \textbf{Exponent} $\begin{prompt}=\answer{5}\end{prompt}$

\end{problem}

\begin{problem} Identify the base and exponent of $\displaystyle \left(\frac{1}{7} \right)^2.$

\hspace{2in} \textbf{Base} $\begin{prompt}=\answer{\frac{1}{7}}\end{prompt}$  \hspace{1in}  \textbf{Exponent} $\begin{prompt}=\answer{2}\end{prompt}$

\end{problem}


\begin{problem} Write the expression $11 \cdot 11 $ using exponential notation.

\begin{multipleChoice}
    \choice{$2 \cdot 11$}
    \choice{$121$}
    \choice[correct]{$11^2$ }
  \end{multipleChoice}

\end{problem}


\begin{problem} Write the expression $9 \cdot 9 \cdot 9 \cdot 9$ using exponential notation.

\begin{multipleChoice}
    \choice{$9 \cdot 4$}
    \choice[correct]{$9^4$ }
    \choice{$9^3$}
  \end{multipleChoice}

\end{problem}


\begin{problem} Write the expression $2^5$ as a product without the exponent.

\begin{multipleChoice}
    \choice[correct]{$2 \cdot 2 \cdot 2 \cdot 2 \cdot 2$}
    \choice{$2 \cdot 5$}
    \choice{$2 \cdot 2 \cdot 2 \cdot 2$ }
  \end{multipleChoice}

\end{problem}


\begin{problem} Write the expression $\left(\displaystyle\frac{1}{5} \right)^3$ as a product without the exponent.

\begin{multipleChoice}
    \choice{$\displaystyle\left(\frac{1}{5} \right) \cdot 3$}
    \choice{$\displaystyle \left(\frac{1}{5} \right) \cdot \left(\frac{1}{5} \right)$ }
    \choice[correct]{$\displaystyle \left(\frac{1}{5} \right) \cdot \left(\frac{1}{5} \right) \cdot \left(\frac{1}{5} \right)$}
  \end{multipleChoice}

\end{problem}


\subsection*{Reading Numbers in Exponential Notation}

Below are some powers of 3 and how they are read in exponential notation.

\fbox{\parbox{6.1in}{
\begin{tabular}{r c c l}
& & \\
3 & = & $3^1$ & ``Three or three to the first power." \\
$3 \cdot 3$ & = & $3^2$ & ``Three to the second power or three \textbf{squared}." \\
$3 \cdot 3 \cdot 3$ & = & $3^3$ & ``Three to the third power or three \textbf{cubed}." \\
$3 \cdot 3 \cdot 3 \cdot 3 $ & = & $3^4$ & ``Three to the fourth power." \\
$3 \cdot 3 \cdot 3 \cdot 3 \cdot 3  $ & = & $3^5$ & ``Three to the fifth power." \\
 & \vdots &  &  \\
\end{tabular}
}}

\begin{problem} Write the expression below in words.

\[  100^3  \]

\begin{multipleChoice}
    \choice{One hundred three}
    \choice{Three hundred.}
    \choice{One hundred times three.}
    \choice[correct]{One hundred cubed.}
  \end{multipleChoice}

\end{problem}

\begin{problem} Write the expression below in words.

\[  10^2 \cdot 3^6 \]

\begin{multipleChoice}
    \choice{Ten times 2 times 3 times 6.}
    \choice[correct]{Ten squared times three to the sixth power.}
    \choice{Thirty to the twelfth power.}
    \choice{Ten to the sixth power times three squared.}
  \end{multipleChoice}

\end{problem}


\subsection*{Positive Powers of 10}

A \textbf{positive power of 10} is a power of 10 in which the exponent is a positive integer: $\{1, 2, 3, 4, \ldots\}.$
Below are the first three positive powers of 10.  \textbf{Notice that the exponent in each power of 10 matches the number of zeros in the result}.  In general, if $n$ is a positive integer, then the value of $10^n$ is the number 1 followed by $n$ zeros.  For example, below you can see that $10^3$ equals the number 1 followed by 3 zeros.
\begin{eqnarray*}
10^1 &=& 10 \ \rightarrow \textrm{1 zero}\\
10^2 &=& 100  \ \rightarrow \textrm{2 zeros} \\
10^3 &=& 1,000 \ \rightarrow \textrm{3 zeros}
%10^4 &=& 10,000
\end{eqnarray*}


\begin{center}
\youtube{ne5Jyg6OK7w}   
\end{center}


\begin{problem} Write the value of $10^4.$


\begin{multipleChoice}
    \choice{$40$}
    \choice[correct]{$10,000$}
    \choice{$1,000$}
    \choice{$40,000$}
  \end{multipleChoice}

\end{problem}

\begin{problem} Write the value of $10^6.$  \textbf{Do not include commas in your answer}. 


\[ 10^6 \begin{prompt}=\answer{1000000}\end{prompt} \]

What number is this?

\begin{multipleChoice}
    \choice{One hundred thousand.}
    \choice[correct]{One million.}
    \choice{One billion.}
    \choice{Ten thousand.}
  \end{multipleChoice}


\end{problem}


\begin{problem} A greedy CEO demanded a pay package worth \$1 trillion.  One trillion equals $10^{12}.$  Write the value of $10^{12}.$


\begin{multipleChoice}
    \choice{$120$}
    \choice{$10,000,000,000$}
    \choice{$1,000,000,000$}
    \choice[correct]{$1,000,000,000,000$}
  \end{multipleChoice}

\end{problem}

%\begin{tikzpicture}[scale=0.7] % Scaling options
    %\chicken[transform shape,3D, body=glaucous] % Customizing with colors
%\end{tikzpicture}

\section*{Fun Fact}

A \textbf{googol} is $10^{100},$ the number 1 followed by 100 zeros!  In 1920, mathematician Edward Kasner was writing a book called \emph{Mathematics and Imagination} and wanted to give a name to $10^{100}.$  He asked his 9-year-old nephew, Milton Sirotta, for a suggestion and little Milton came up with the word googol.  Decades later, it inspired the name of one of the biggest tech companies, \textcolor{googleblue}{\textbf{G}}%
\textcolor{googlered}{\textbf{o}}%
\textcolor{googleyellow}{\textbf{o}}%
\textcolor{googleblue}{\textbf{g}}%
\textcolor{googlegreen}{\textbf{l}}%
\textcolor{googlered}{\textbf{e}}.

\subsection*{Multiplying and Dividing by Positive Powers of 10}

When we multiply or divide a number by a positive power of 10, there is a relationship between the movement of the decimal point and the number of zeros
in the power of 10.  \textbf{Once you notice this relationship, you won't need to perform multiplication or long division to find the answer for these types of problems}.
Look for the relationship between the number of zeros in the power of 10 and the movement of the decimal point in the multiplication problems shown below.
\begin{eqnarray*}
0.625 (10) &=& 6.25 \hspace{0.2in} \textrm{The decimal point moved 1 place to the right.} \\
0.625 (100) &=&  62.5  \hspace{0.2in} \textrm{The decimal point moved 2 places to the right.} \\
0.625 (1000) &=& 625 \hspace{0.25in} \textrm{The decimal point moved 3 places to the right.}\\
& & \\
5 (10) &=& 50 \hspace{0.34in} \textrm{The decimal point moved 1 place to the right.} \\
5 (100) &=&  500  \hspace{0.27in} \textrm{The decimal point moved 2 places to the right.} \\
5 (1000) &=& 5000 \hspace{0.2in} \textrm{The decimal point moved 3 places to the right.}
\end{eqnarray*}

\begin{problem} Write a general rule for multiplying a number by a positive power of 10.


\begin{multipleChoice}
    \choice{To multiply a number by a positive power of 10, count the number of zeros in the power of 10, then move the decimal point to the \textbf{left} that many places.}
    \choice[correct]{To multiply a number by a positive power of 10, count the number of zeros in the power of 10, then move the decimal point to the \textbf{right} that many places.}
  \end{multipleChoice}

\end{problem}


\begin{center}
\youtube{CTfY61uLv80}
\end{center}

\begin{problem}  Use the rule from the previous problem to find each answer.

\[ 2.8 (10) \begin{prompt}=\answer{28}\end{prompt} \]


\[ 3.0525 (1000) \begin{prompt}=\answer{3052.5}\end{prompt} \]

\[ 620 (100) \begin{prompt}=\answer{62,000}\end{prompt} \]

\end{problem}


Look for the relationship between the number of zeros in the power of 10 and the movement of the decimal point in the division problems shown below.

\begin{eqnarray*}
\frac{43.2}{10} &=&  4.32 \hspace{0.48in} \textrm{The decimal point moved 1 place to the left.} \\
& & \\
\frac{43.2}{100} &=& 0.432 \hspace{0.42in}  \textrm{The decimal point moved 2 places to the left.} \\
& & \\
\frac{43.2}{1000} &=& 0.0432 \hspace{0.365in}  \textrm{The decimal point moved 3 places to the left.} \\
& & \\
& & \\
\frac{3}{10} &=& 0.3 \hspace{0.42in} \textrm{The decimal point moved 1 place to the left.} \\
& & \\
\frac{3}{100} &=& 0.03 \hspace{0.36in} \textrm{The decimal point moved 2 places to the left.} \\
& & \\
\frac{3}{1000} &=& 0.003 \hspace{0.3in} \textrm{The decimal point moved 3 places to the left.} \\
\end{eqnarray*}

\begin{problem} Write a general rule for dividing a number by a positive power of 10.


\begin{multipleChoice}
    \choice[correct]{To divide a number by a positive power of 10, count the number of zeros in the power of 10, then move the decimal point to the \textbf{left} that many places.}
    \choice{To divide a number by a positive power of 10, count the number of zeros in the power of 10, then move the decimal point to the \textbf{right} that many places.}
  \end{multipleChoice}

\end{problem}

\begin{problem}  Use the rule from the previous problem to find each answer.


\[ \frac{63}{1000} \begin{prompt}=\answer{0.063}\end{prompt} \]

\[ \frac{35.5}{10} \begin{prompt}=\answer{3.55}\end{prompt} \]

\[ \frac{1}{100} \begin{prompt}=\answer{0.01}\end{prompt} \]

\end{problem}

%\hspace{4.3in} 
%\begin{tikzpicture}[scale=0.6] % Scaling options
%    \owl[transform shape,3D, body=glaucous, graduate=black, tassel=gray] % Customizing with colors
%\end{tikzpicture}


\section*{Summary}


\normalsize
\begin{enumerate}
\item \textbf{Exponential Notation}

Exponential notation is shorthand for repeated multiplication of the same quantity.  A number written in exponential notation has a base and an exponent.  The exponent indicates the number of times to multiply the base.  For example, the product $8 \cdot 8 \cdot 8$ is written as $8^3$ in exponential notation.  The number 8 is the base and 
the number 3 is the exponent. 

\item \textbf{Positive Powers of 10}

A \textbf{positive power of 10} is a power of 10 in which the exponent is a positive integer: $\{1, 2, 3, 4, \ldots \}.$  Below are the first four positive powers of 10.
\begin{eqnarray*}
10^1 &=& 10 \\
10^2 &=& 100   \\
10^3 &=& 1,000 \\
10^4 &=& 10,000
\end{eqnarray*}

In general, if $n$ is a positive integer, then the value of $10^n$ is the number 1 followed by $n$ zeros.

\item \textbf{Multiplying a Number by a Positive Power of 10}

To multiply a number by a positive power of 10, count the number of zeros in the power of 10, then move the decimal point to the \textbf{right} that many places.


\item \textbf{Dividing a Number by a Positive Power of 10}

To divide a number by a positive power of 10, count the number of zeros in the power of 10, then move the decimal point to the \textbf{left} that many places.



\end{enumerate}



\end{document}
