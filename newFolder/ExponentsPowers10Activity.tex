\documentclass{ximera}

\title{Exponents and Positive Powers of 10}   % Use xmlatex all -s to compile when SERVE refuses to work.
\author{Kelly Stady}
\license{CC: 0}         % replace with an appropriate license, or set it in xmPreamble

\begin{document}
\begin{abstract}

\end{abstract}
\maketitle
\label{xim:ExponentsPowers10Activity}


\section*{Exponential Notation}

Exponential notation is shorthand for repeated multiplication of
the same quantity. For example, the product $5 \cdot 5 \cdot 5$ can be rewritten more simply
as $5^3.$  In this example, the number 5 is called the
\textbf{base} and the number 3 is called the \textbf{exponent}. In
general, the exponent indicates the number of times to multiply
the base.


\begin{problem} Identify the base and exponent of $10^7.$


\textbf{Base} $\begin{prompt}=\answer{10}\end{prompt}$  \qquad  \textbf{Exponent} $\begin{prompt}=\answer{7}\end{prompt}$


\end{problem}

\begin{problem} Write the expression $11 \cdot 11 $ using exponential notation.

\begin{multipleChoice}
    \choice{$2 \cdot 11$}
    \choice{$121$}
    \choice[correct]{$11^2$ }
  \end{multipleChoice}

\end{problem}

\begin{problem} Write the expression $3^4$ as a product without the exponent.

\begin{multipleChoice}
    \choice[correct]{$3 \cdot 3 \cdot 3 \cdot 3$}
    \choice{$3 \cdot 4$}
    \choice{$4 \cdot 4 \cdot 4$ }
  \end{multipleChoice}

\end{problem}

\subsection*{Reading Numbers in Exponential Notation}

Below are some powers of 3 and how they are read in exponential notation.


\fbox{\parbox{6.1in}{
\begin{tabular}{r c c l}
& & \\
3 & = & $3^1$ & ``Three or three to the first power." \\
$3 \cdot 3$ & = & $3^2$ & ``Three to the second power or three \textbf{squared}." \\
$3 \cdot 3 \cdot 3$ & = & $3^3$ & ``Three to the third power or three \textbf{cubed}." \\
$3 \cdot 3 \cdot 3 \cdot 3 $ & = & $3^4$ & ``Three to the fourth power." \\
$3 \cdot 3 \cdot 3 \cdot 3 \cdot 3  $ & = & $3^5$ & ``Three to the fifth power." \\
 & \vdots &  &  \\
\end{tabular}
}}

\begin{problem} Write the product below in words.

\[  10^3 \cdot 2^5 \]

\begin{multipleChoice}
    \choice{Ten times 3 times 2 times 5.}
    \choice[correct]{Ten cubed times two to the fifth power.}
    \choice{Three to the tenth power times five squared.}
  \end{multipleChoice}

\end{problem}


\subsection*{Positive Powers of 10}

Below are the first three positive powers of 10.  Notice that the exponent of 10 corresponds to the number of zeros in the result.  In general, if $n$ is a positive whole number, then  $10^n$ is equal to the number 1 followed by $n$ zeros.  For example, in the table below you can see that $10^3$ equals the number 1 followed by 3 zeros.
\begin{eqnarray*}
10^1 &=& 10 \\
10^2 &=& 100  \\
10^3 &=& 1,000
%10^4 &=& 10,000
\end{eqnarray*}


\begin{problem} Use what you learned above to write the value of $10^4.$


\begin{multipleChoice}
    \choice{$40$}
    \choice[correct]{$10,000$}
    \choice{$1,000$}
  \end{multipleChoice}

\end{problem}


\begin{problem} A greedy CEO demanded a pay package worth \$1 trillion.  One trillion is equal to $10^{12}.$  Write the value of $10^{12}.$


\begin{multipleChoice}
    \choice{$120$}
    \choice{$10,000,000,000$}
    \choice{$1,000,000,000$}
    \choice[correct]{$1,000,000,000,000$}
  \end{multipleChoice}

\end{problem}

\subsection*{Multiplying and Dividing by Positive Powers of 10}

When we multiply or divide a number by a positive power of 10, there is a relationship between the movement of the decimal point and the number of zeros
in the power of 10.


Consider the multiplication problems shown below.

\begin{eqnarray*}
0.625 (10) &=& 6.25 \hspace{0.2in} \textrm{The decimal point moved 1 place to the right.} \\
0.625 (100) &=&  62.5  \hspace{0.2in} \textrm{The decimal point moved 2 places to the right.} \\
0.625 (1000) &=& 625 \hspace{0.25in} \textrm{The decimal point moved 3 places to the right.}\\
& & \\
5 (10) &=& 50 \hspace{0.34in} \textrm{The decimal point moved 1 place to the right.} \\
5 (100) &=&  500  \hspace{0.27in} \textrm{The decimal point moved 2 places to the right.} \\
5 (1000) &=& 5000 \hspace{0.2in} \textrm{The decimal point moved 3 places to the right.}
\end{eqnarray*}

\begin{problem} Use the pattern seen above to write a general rule for multiplying a number by a positive power of 10.


\begin{multipleChoice}
    \choice{To multiply a number by a positive power of 10, move the decimal point to the left the same number of places as there are zeros in the power of 10.}
    \choice[correct]{To multiply a number by a positive power of 10, move the decimal point to the right the same number of places as there are zeros in the power of 10.}
  \end{multipleChoice}

\end{problem}

Consider the division problems shown below.

\begin{eqnarray*}
\frac{43.2}{10} &=&  4.32 \hspace{0.48in} \textrm{The decimal point moved 1 place to the left.} \\
& & \\
\frac{43.2}{100} &=& 0.432 \hspace{0.42in}  \textrm{The decimal point moved 2 places to the left.} \\
& & \\
\frac{43.2}{1000} &=& 0.0432 \hspace{0.365in}  \textrm{The decimal point moved 3 places to the left.} \\
& & \\
& & \\
\frac{3}{10} &=& 0.3 \hspace{0.42in} \textrm{The decimal point moved 1 place to the left.} \\
& & \\
\frac{3}{100} &=& 0.03 \hspace{0.36in} \textrm{The decimal point moved 2 places to the left.} \\
& & \\
\frac{3}{1000} &=& 0.003 \hspace{0.3in} \textrm{The decimal point moved 3 places to the left.} \\
\end{eqnarray*}

\begin{problem} Use the pattern seen above to write a general rule for dividing a number by a positive power of 10.


\begin{multipleChoice}
    \choice[correct]{To divide a number by a positive power of 10, move the decimal point to the left the same number of places as there are zeros in the power of 10.}
    \choice{To divide a number by a positive power of 10, move the decimal point to the right the same number of places as there are zeros in the power of 10.}
  \end{multipleChoice}

\end{problem}



\end{document}
